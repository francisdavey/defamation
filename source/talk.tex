\documentclass[]{article}
\usepackage{lmodern}
\usepackage{amssymb,amsmath}
\usepackage{fixltx2e} % provides \textsubscript

\usepackage{mathspec}
\usepackage{xltxtra,xunicode}
\defaultfontfeatures{Mapping=tex-text,Scale=MatchLowercase}
\newcommand{\euro}{€}

% use upquote if available, for straight quotes in verbatim environments
\IfFileExists{upquote.sty}{\usepackage{upquote}}{}
% use microtype if available
\IfFileExists{microtype.sty}{%
\usepackage{microtype}
\UseMicrotypeSet[protrusion]{basicmath} % disable protrusion for tt fonts
}{}

\usepackage[setpagesize=false, % page size defined by xetex
              unicode=false, % unicode breaks when used with xetex
              xetex]{hyperref}
\hypersetup{breaklinks=true,
            bookmarks=true,
            pdfauthor={},
            pdftitle={},
            colorlinks=true,
            citecolor=blue,
            urlcolor=blue,
            linkcolor=magenta,
            pdfborder={0 0 0}}
\urlstyle{same}  % don't use monospace font for urls
\setlength{\parindent}{0pt}
\setlength{\parskip}{6pt plus 2pt minus 1pt}
\setlength{\emergencystretch}{3em}  % prevent overfull lines
\setcounter{secnumdepth}{0}

\usepackage{titlesec}

\date{}

\begin{document}

\subsection{Introduction}

The purpose of this training session is to give you a basic
understanding of the way in which defamation law operates. It is aimed
to make it easier for you to understand the kinds of problems that
running a website might create concerning defamation law and to
understand any legal advice that is given to you.

\section{I - the law of defamation}

\subsection{Terminology}

\subsubsection{\texorpdfstring{``defamation''}{defamation}}

These notes will start with an explanation of what is and is not ``defamatory'', but first something needs to be got out of the way. Sometimes people use the term ``defamatory'' only for statements that are untrue. Indeed there can be quite heated arguments about it. As will, I hope, be clear, it is better to think of ``defamatory'' as referring to certain kinds of statements that hurt someone's reputation even if they might be quite true. 

In other words the statement:

``Lord Archer is a convicted criminal''

is defamatory (but true).

\subsubsection{\texorpdfstring{``libel'' and
``slander''}{libel and slander}}

Defamation is an umbrella term for two distinct kinds of claim
(technically known as \textbf{causes of action}{), which are:}

\begin{itemize}
\item
  \textbf{libel}{ - written, broadcast or performed statements}
\item
  \textbf{slander}{ - spoken (or otherwise impermanent) statements}
\end{itemize}

Since defamatory material published on websites will be libel, the practical differences between libel and slander are left to Appendix~\ref{sec:applibelvslander}.

\subsection{Elements of defamation}

To sue successfully for defamation C will need to prove:

\begin{enumerate}
\item  A statement S
\item  Identifiably about C
\item  S has the meaning M
\item  M is defamatory of C
\item  S was published by D
\item The publication of S has caused or is likely to cause {\bf serious harm} to the reputation of C
\end{enumerate}

\subsection{Serious harm}
The requirement of serious harm was added by the \href{http://www.legislation.gov.uk/ukpga/2013/26/contents/enacted}{Defamation Act 2013}.\footnote{s1} There has been very little consideration of what it means by the courts, but it is likely to make life more difficult for claimants.

A claimant will have to prove that their reputation either has suffered serious harm or that it is likely to do so. There will be some cases where serious harm is obvious (such as alleging that someone is a terrorist\footnote{{\protect\it Cooke v MGN} \href{http://www.bailii.org/ew/cases/EWHC/QB/2014/2831.html}{[2014] EWHC 2831 (QB)} [43]}) but in many cases the claimant will have to produce some evidence of harm to their reputation or risk having the claim dismissed at an early stage.

In the first case to consider what ``serious harm'' meant,\footnote{{\protect\it Cooke v MGN} {[2014] EWHC 2831 (QB)}} the defendant newspaper had published an apology for their defamatory article. The judge decided that meant that there was unlikely to be any risk of future harm -- because anyone searching the internet for the article would almost certainly find the apology -- and since there was no proof that any harm had been suffered, the claim was dismissed.

{\it Cooke} established that the time at which to test whether serious harm had taken place (or was likely to take place) was probably the date of issue (i.e. when the claim was started) rather than before. This seems to mean that a newspaper that is threatened with a libel claim can quickly apologies and avoid paying damages, although they may have to pay the claimant's costs of putting together their complaint.

It is far from clear exactly what you would need to do in order to prove serious harm.

For a ``body that trades for profit'', ``serious harm'' requires that the body has suffered ``serious financial loss''.\footnote{s1(2)} I hope that this will make it more difficult for a commercial organisations to sue individuals over trivial alleged libels, such as poor reviews, as, for example, Pimlico Plumbers threatened to do last year over reviews on Yelp.

\subsection{Some anatomy of a defamation claim}

Defamation is part of an area of law known as the ``law of
\textbf{tort}{'' from which we derive two eye-watering words:
}\textbf{tortious} {and }\textbf{tortfeasor}{. The following is an
outline of how a defamation (and indeed most tort claims) is carried
out. I am simplifying by omitting applications and injunctions which can
happen }\emph{{at any time.}}

If someone feels they have been defamed they will do the following:

\begin{enumerate}
\item
  {They should (on pain of costs penalties) carry out the
  }\textbf{pre-action protocol}{ requiring them to write a letter before
  action to the person/people they intend to sue.}
\item
  {They will fill in a }\textbf{claim form}{ (old word ``writ'') and ask
  a court to }\textbf{issue}{ it. }
\item
  {They are now know as the }\textbf{claimant} {(old word} \textbf{}
  {``}\textbf{plaintiff}{''} \textbf{} {in Scotland
  ``}\textbf{pursuer}{'') - we will call them ``C'' throughout.}
\item
  {Either with the claim form, or shortly afterwards, the claimant must
  also serve what is known as a }\textbf{particulars of claim}{ - which
  sets out the details of the claim which they are bringing.}
\item
  {They will then }\textbf{serve}{ it (i.e. give it to one way or
  another) on one or more people who will then be }\textbf{defendants}{
  (affectionately ``D'')}
\item
  The defendants will then create a document known as a defence which
  will do two things:

  \begin{enumerate}
  \item
    deal with each point in the particulars of claim saying whether the
    defendant denies it, agrees that it is true, or does not know;
  \item
    sets out the defendant's positive defences and any facts that the
    defendant relies on.
  \end{enumerate}
\item
  Many things can happen: the court will have one or more case
  management conferences or hearings to decide how the case should
  proceed and when evidence should be served; either party may make
  applications (and they usually will as we shall see), injunctions may
  be granted or discharged and so on\ldots{}.
\item
  {Eventually the parties will have exchanged }\textbf{witness
  statements}{ (the written evidence the witness is to give) and then
  there will be a}
\item
  Trial (usually before a jury)
\item
  After the trial if the claimant has one they may seek to enforce any
  judgment; there may be an appeal etc.
\end{enumerate}

\subsection{Applications and injunctions}

{In the civil procedure rules anyone can make what is known as an
``application'' at any
time}{{.\hyperref[sdfootnote1sym]{\textsuperscript{1}} Applications are
very flexible. In rare circumstances:}}

\begin{itemize}
\item
  they can be made even before procedings have started
\item
  they can be made by non-parties.
\end{itemize}

In defamation proceedings the applications one is likely to see are:

\begin{itemize}
\item
  {{for an order to stop publication (called an
  }}{\textbf{injunction}}{{), perhaps before publication has happened
  and even without the knowledge (at first) of the defendant}}
\item
  {{to find out information, particularly the identity of a proposed
  defendant, from an innocent third party - a }}{\textbf{Norwich
  Pharmacal} }{{order.}}
\item
  to ask the judge to decide a question of law
\item
  {{for summary judgment (on the grounds that the other party has no
  }}\emph{{reasonable prospect of success}}{{)}}
\item
  for striking out (i.e. a deletion of part of one sides pleadings on
  the ground that they are irrelevant, wrong in law or disclose no cause
  of action)
\item
  \ldots{} a myriad of case management orders
\end{itemize}

\subsection{Judges and juries}

{Fox's Libel Act\hyperref[sdfootnote2sym]{\textsuperscript{2}} reformed
the criminal procedure for libel so that trial was always before a jury.
That rule has been adapted to civil proceedings so that a defamation
claim is almost always heard in a trial by jury. This is now unusual
because in England and Wales we have largely dropped the use of juries
for civil claims except for a small number of rather special causes of
action (such as claims against the Police for false imprisonment).
Because trial is usually by jury it is important to understand the
different roles judges and juries have.}

\begin{itemize}
\item
  Judges - decide questions of law, which can set precedent
\item
  Juries - decide questions of fact, which cannot set precedent
\end{itemize}

{Most lawyers understand the law/fact division but it is not at all
plain to the lay person. For example whether a statement }\textbf{is}
{defamatory is obviously a question of fact (which a jury would decide)
but whether it is the kind of statement that }\textbf{could be}{
defamatory is a legal question (which a judge would decide).}

This has several consequences:

\begin{enumerate}
\item
  {The judge can make decisions about what meanings a statement could
  have and whether they could be defamatory at any stage prior to the
  trial as well as at the trial. A common tactic to use at an early
  stage is to try to knock out meanings or indeed the whole case by an
  application for }\textbf{summary judgment}{ or }\textbf{striking
  out}{.}
\item
  {The case law has to be read with care because the decisions are
  usually about what }\emph{{could}}{{ be defamatory not what
  }}\emph{{was}}{{ defamatory. The decision of a jury is neither here
  nor there.}}
\end{enumerate}

This means that in a defamation claim the following processes could
occur.

\begin{enumerate}
\item
  C pleads that the S means M
\item
  The judge considers:

  \begin{enumerate}
  \item
    could S mean M? (is M a meaning S is capable of bearing)
  \item
    is M capable of being defamatory?
  \end{enumerate}
\item
  The jury considers:

  \begin{enumerate}
  \item
    what does S mean?
  \item
    {is }\textbf{that}{ meaning defamatory?}
  \end{enumerate}
\end{enumerate}

\subsection{What is defamatory?}

The courts have made a number of efforts to define what it is for a statement to be ``defamatory''. In \href{http://www.bailii.org/ew/cases/EWHC/QB/2010/1414.html}{\it Thornton v Telegraph Media Group}\footnote{[2010] EWHC 1414 (QB)}, the judge listed 9 well-known examples, including

\begin{quote}
... words [that] tend to lower the plaintiff in the estimation of right-thinking members of society generally?\footnote{{\it Sim v Stretch} [1936] 2 All ER 1237}
\end{quote}

That expression would seem to apply only where the claimant has done something wrong. Surely ``right-thinking members of society'' ought not to treat someone less well because of something that was not their moral fault. However, the courts have held that defamation does include statements that carry no moral criticism, for example that they have a disease.

For example in {\it Youssoupoff v Metro-Goldwyn-Mayer Pictures} (1934) 50 TLR 581 the court held that an allegation that someone had been raped was defamatory. To some extent this may reflect prevailing social attitudes in the 1930's. The court held that a statement is also defamatory if it ``tends to make the plaintiff be shunned and avoided and that without any moral discredit on [the plaintiff's] part'', which would include an allegation that they were insane or had an infectious disease.

In some cases, a statement that held someone up to ``contempt, scorn or ridicule'' has been accepted as defamatory. In {\it Berkhoff v Birchill}, Stephen Berkhoff sued Julie Birchill for implying that he was ``hideously ugly'' in a  ridicule alone might be defamatory

\begin{itemize}
\item
  likely effect of the words (not the actual effect)
\item
  on ``right thinking persons generally'' -- see Byrne v Deane for an
  example of something that was not defamatory because it claimed
  someone had acted lawfully.
\end{itemize}

\subsection{Examples of imputations}

A useful exercise is to think through the following imputations and
consider whether they are (or are not) defamatory.

\begin{enumerate}
\item
  C is insane
\item
  {C has HIV (does it make a difference whether C is ``innocent'' or
  contracted HIV through promiscuous }\emph{{gay sex}}{{)?}}
\item
  C has been raped
\item
  C has/had heart disease (what about 'flu?)
\item
  C is illegitimate
\item
  C has leprosy
\item
  X is a better journalist than C
\item
  C is a lawyer of only average ability
\end{enumerate}

\subsection{Meaning}

{In deciding what something means (or what meanings it }\emph{{could
}}{{have) the courts have developed a variety of approaches and rules.}}

\begin{itemize}
\item
  only one meaning - even though a statement may have multiple meanings,
  juries are directed to determine ``the'' meaning of a statement.
\item
  intention is irrelevant (though it can be relevant to the Defamation
  Act 1996 and to some defences)
\end{itemize}

\subsubsection{\texorpdfstring{``natural and
ordinary''}{natural and ordinary}}

There are in fact two kinds of meaning that can be pleaded by a
claimant. The first is what is known as the ``natural and ordinary''
meaning of the words.

\begin{itemize}
\item
  natural and ordinary meaning includes imputations and inferences

  \begin{itemize}
  \item
    ``{Have you heard that Fox was reported twice as a spy'' -
    defamatory implication that he was
    guilty.\hyperref[sdfootnote3sym]{\textsuperscript{3}}}
  \end{itemize}
\item
  hearer must be reasonably justified in understanding words to be
  defamatory
\item
  not strained, forced or utterly unreasonable

  \begin{itemize}
  \item
    {not enough that it }\emph{{might}}{{ be understood in a defamatory
    sense (see Capital and Counties Bank v Henty)}}
  \item
    ``{{suspicious people might get a defamatory meaning out of `chop
    and tomato sauce'''\hyperref[sdfootnote4sym]{\textsuperscript{4}}}}
  \end{itemize}
\item
  The ordinary reasonable fair-minded reader has been constructed by the
  courts as the person from whose point of view the meaning is to be
  assessed. Some of the qualities such a person is thought to have are:

  \begin{itemize}
  \item
    reasonable intelligence
  \item
    ordinary person's general knowledge
  \item
    may include implications and inferences
  \item
    fair minded and reasonable
  \item
    may be guilty of a certain amount of loose thinking
  \item
    does not read a sensational article with cautious and critical care
  \item
    goes by broad impression
  \item
    does not construe words as would a lawyer
  \end{itemize}
\end{itemize}

\subsubsection{\texorpdfstring{``Legal'' Innuendo
meanings}{Legal Innuendo meanings}}

The word ``innuendo'' is used in two ways by defamation lawyers. The
first in a non-technical sense to mean an inferred or implied meaning
into words, the second is the label attached to a special kind of
meaning that can be pleaded by a claimant. To distinguish it from the
normal use of the word it is often called the ``true'' or ``legal''
innuendo meaning.

A legal innuendo is a meaning that a statement possesses because of the
knowledge of additional facts that are not general knowledge. It may be
that only some people to whom the statement is published are aware of
those additional facts (and the claimant will have to prove it). That
additional knowledge can make something that looked innocent into a
defamation and vice versa.

A legal innuendo gives rise to a separate cause of action. I can sue
once on publication to the general public and another time on
publication to those with special knowledge.

\subsection{Defences}

Obviously a defendant can defend themselves by denying what the claimant
claims (eg by denying that the statement was defamatory or was even
published). But the publication of a defamatory statement can be defended in certain circumstances:

\begin{itemize}
\item
  Truth -- the defendant proves the imputation conveyed by the statement is substantially true\footnote{s2, Defamation Act 2013}
\item
  Privilege -- otherwise known as ``absolute'' privilege: the defendant
  proves that the statement was made on a privileged occasion. For
  example

  \begin{itemize}
  \item
    reports of a statement made in Parliament
  \item
    contemporaneous reporting of legal proceedings
  \item
    statements made in legal proceedings
  \item
    (possibly) statements made to one's lawyer in obtaining legal advice
    (there is some doubt over this - it may just be a qualified
    privilege.
  \end{itemize}
\item
  Qualified privilege
\item Honest opinion
\item
  Offer of amends
\end{itemize}

{{Qualified privilege and fair comment are defeated by the claimant
proving }}{\textbf{malice}}{{. Like everything else in defamation
``malice'' has a peculiar technical meaning. Simply put it implies
either that the statement was made for an improper motive or in the
absence of honest belief in its truth.}}

\subsection{Truth}

It is a defence to prove that the ``imputation'' conveyed by the defamatory statement is true.

At common law there was a serious limitation on this defence comes in the form of the {\bf repetition rule}. Where a defendant repeats a statement
made by another it is not enough to prove that the statement was made, but that the statement was true. 

For example even if the statement:

``There is a rumour that C murdered X''

were literally be true, the defendant would usually have to prove that C did murder X to defend a libel claim.

In {\it Lewis v Daily Telegraph Ltd}\href{}{[1964] A.C. 234} Lord Devlin observed\foonote{at p.283--4 } that:

``\ldots{} you cannot escape liability for defamation by putting the
libel behind a prefix such as `I have been told that \ldots{}' or `It is
rumoured that \ldots{}' and then asserting that it was true that you had
been told or that it was in fact being rumoured. You have \ldots{} to
prove that the subject matter of the rumour was true.''

this means that:

\begin{itemize}
\item
  adding ``allegedly'' after a possibly defamatory quote from someone
  else does no good at at all
\item
  publishing a response by the defamed person, or noting that something
  is merely someone else's opinion is no protection.
\end{itemize}

Sometimes a statement that implies wrongdoing by C may means something a little weaker than that C is actually guilty of the wrong. Consider:

``Officers of the City of London Fraud Squad are inquiring into the
affairs of Rubber Improvement Ltd. and its subsidiary companies. The
investigation was requested after criticisms of the chairman's statement
and the accounts by a shareholder at the recent company meeting.''

{{What could that mean? Its direct meaning is easily justifiable (the
Fraud Squad were investigating) but is that all? The Plaintiff in the
case pleaded that it meant that they were guilty of fraud. The House of
Lords decided that was going too far, but that it }}\emph{{could
}}{{mean that they had so conducted themselves as to attract
suspicion.\hyperref[sdfootnote5sym]{\textsuperscript{5}}}}

{{The courts have developed what are called the three }}\emph{{Chase}}{{
levels of meaning:\hyperref[sdfootnote6sym]{\textsuperscript{6}}}}

\begin{enumerate}
\item
  guilty
\item
  {{reasonable grounds for suspicion of guilt - must be based on the
  conduct of the claimant (the }}{\textbf{conduct rule}}{{) and cannot
  be based merely on the fact that the police decided to investigate}}
\item
  reasonable grounds to investigate guilt
\end{enumerate}

What is unclear is whether the ``repetition rule'' or the division of meanings into {\it Chase} levels of meaning has been abolished by the Defamation Act 2013.

\subsection{Qualified Privilege}

\begin{itemize}
\item
  Co-ordination of duty and interest
  \begin{itemize}
  \item
    D had a duty or interest in publishing the statement
  \item
    the recipient of the publication had a duty or interest in receiving it
  \end{itemize}
\item
  Defeated by malice
\item   Examples
  \begin{itemize}
  \item confidential references
  \item communications amongst the team
  \end{itemize}
\end{itemize}

\subsection{Honest Opinion}
This defence seems to be intended to allow people to express honestly held opinions. It applies if three conditions hold:

\begin{itemize}
\item the statement was a statement of opinion 
\item the basis of that opinion was indicated
\item an honest person could have held the opinion, on the basis of
  \begin{itemize}
  \item any fact which existed at the time of publication
  \item anything asserted in a privileged statement
    \begin{itemize}
    \item matter of public interest
    \item peer-reviewed statement in scientific or academic journal
    \item report of court proceedings
    \item various other statutory privileged reports
    \end{itemize}
  \end{itemize}

\end{itemize}

The Claimant can defeat defence if they prove that D did not hold the opinion, except where D published someone else's statement. In that case the Claimant can defeat the defence if they show that the defendant knew, or ought to have known, that the original author of the statement did not hold the opinion.

\subsection{Outcomes}

Why be afraid of a defamation claim?

\begin{itemize}
\item
  Injunction

  \begin{itemize}
  \item
    May be made in the interim
  \item
    Balance of convenience test for interim injunctions
  \end{itemize}
\item
  Damages

  \begin{itemize}
  \item
    Potentially very large
  \item
    Decided by juries, although the Court of Appeal does exercise some
    control over it. Juries sounds all very wonderful in theory but they
    have absolutely no sense of proportion when it comes to damages
  \end{itemize}
\item
  {{Costs - also potentially vast since defamation cases are usually
  heard in the High Court, are legally complex, dealt with by a very
  small group of specialised lawyers and involve long trials by jury. In
  general defamation claims are }}\emph{{very }}{{attractive to lawyers
  who can laugh all the way to the bank.}}
\end{itemize}

An example of excessive awards is a case where an article claiming Elton
John had a habit of chewing but not swallowing (in fact spitting out)
food; had been observed doing it and medical evidence suggested this was
a sign of Bulimia. The jury awarded compensatory damages of £75,000
(reduced by the Court of Appeal to £25,000).

\subsection{Offer of Amends}

\begin{itemize}
\item
  An offer to

  \begin{itemize}
  \item
    make and publish:

    \begin{itemize}
    \item
      a suitable correction; and
    \item
      a sufficient apology
    \end{itemize}
  \item
    pay compensation to be agreed or determined
  \end{itemize}
\item
  Plus

  \begin{itemize}
  \item
    not an admission of liability
  \item
    acceptance prevents future claim
  \end{itemize}
\item
  Minus

  \begin{itemize}
  \item
    only useful for innocent defamations
  \item
    may not use another defence
  \end{itemize}
\end{itemize}

\section{II - Issues relevant to websites}

There are several, overlapping, defences available to a web host that may act as a defence against a claim for libel. They are:

\begin{itemize}
\item Innocent dissemination
\item Section 1, Defamation Act 1996
\item Section 10, Defamation Act 2013
\item Section 5, Defamation Act 2013
\item E-commerce directive
\end{itemize}

\subsection{Defamation Act 1996}

Under section 1 of the Defamation Act 1996, a defendant is not liable
for a statement if they:

\begin{itemize}
\item were not the author, editor or publisher
\item took reasonable care in respect of publication
\item did not know and had no reason to know that D caused or contributed to
  the publication of a defamatory statement
\end{itemize}

``Author'', straightforwardly means the person who originated the statement. 

An ``editor'' is someone who has editorial or equivalent responsibility for the content of the statement or the decision to publish it. This is likely to include an activity such as moderating comments pre-publication, but I do not think it would include moderating comments in response to complaints (eg of abuse). 

The word ``publisher'', in the context of the 1996 Act (and section 10 of the Defamation Act 2013, see below) does not mean simply someone who publishes (in the common law sense) a statement. It means a person whose business is issuing material to the public (or a section of the public) and who issues material containing the statement in the course of that business.

The key word here is ``issue''. For example, a bookseller is not a ``publisher'' for the purposes of this definition because they merely distribute books, rather than issuing them to the public. 

In {\it McGrath v Dawkins} \href{http://www.bailii.org/ew/cases/EWHC/QB/2012/B3.html}{[2012] EWHC B3}, the court rules that Amazon was not a ``commercial publisher'' of its website, because it the website was not its main source of revenue which comes from selling books etc. That seems to me to be rather generous to Amazon. The court also held that, despite the fact that Amazon did look at reviews and comments in response to complaints, it was not an ``editor''.

\subsection{Section 10, Defamation Act 2013}
Like section 1 of the 1996 Act, Section 10 of the 2013 Act is a generic defence to defamation that is not restricted to internet publication, but it will be generally useful for internet publishers.

Where a claim is for defamation is brought against someone who is not the author, editor or publisher (in the sense of the 1996 Act discussed above), the court has no jurisdiction to hear it unless the court is satisfied that it is not reasonably practicable for an action to be brought against the author, editor or publisher.

In practice this means that a claimant suing someone who was not an author, editor or publisher, would almost certainly have to produce evidence that it was ``not reasonably practicable'' to sue that actual author etc. There is, as yet, no case law on what this means, so there are several points that are not clear.

In the first place, it is possible to sue someone without knowing their identity. In {\it Bloomsbury Publishing Group v News Group Newspapers}\footnote{\href{http://www.bailii.org/ew/cases/EWHC/Ch/2003/1205.html}{[2003] EWHC 1205 (Ch)}} the court allowed a claim to proceed where instead of naming the defendants, they were described as:

\begin{quote}
  the person or persons who have offered the publishers of the Sun, the Daily Mail, and the Daily Mirror newspapers a copy of the book 'Harry Potter and the order of the Phoenix' by JKRowling or any part thereof and the person or persons who has or have physical possession of a copy of the said book or any part thereof without the consent of the claimants.
\end{quote}

This approach has been used on a number of other occasions and has been endorsed by the Supreme Court.\footnote{Secretary of State for Environment, Food and Rural Affairs \href{http://www.bailii.org/uk/cases/UKSC/2009/11.html}{[2009] UKSC 11}} Anonymity does not make {\it starting} a claim very much more difficult. 

Serving the claim on the defendant might require a little more effort than usual. If it is clear that they will read something (eg twitter, facebook, a blog post) then a court might permit service via tweet or post. If their email address is available, then service by email could be used. A web host could be required to supply that email address by a Norwich Pharmacal Order.

It is quite possible that section 10 requires not that {\it starting} a claim be reasonably practicable, but the claim itself. Some lawyers have suggested that suing someone in the United States, where English defamation judgments are hard to enforce, might not be ``reasonably practicable'' because the claim itself would be pointless. This is not the literal meaning of section 10, but it is entirely possible that the courts will decide to interpret it that way.

What this means is that identity is not the end of the story. But if you are threatened over defamatory postings where you are not the ``author, editor or publisher'' and the real author etc is easily identifiable, you have a good argument that you cannot be sued.

\subsection{Section 5, Defamation Act 2013}
This is a defence available to ``operators of websites'' for statements posted on their website, provided they did not post the statement themselves -- and presumably where it was posted on the their behalf either. A claimant may be able to defeat the defence, but only if they are able to prove three things:

\begin{itemize}
\item it was not possible for them to identify the person who posted the statement
\item they gave a ``notice of complaint'' to the operator
\item the operator failed to respond to the notice correctly
\end{itemize}

Section 5 goes on to say that it is only possible to ``identify'' the poster if the claimant has sufficient information to bring proceedings against them. See my remarks on section 10 for difficulties with that condition.

The idea of section 5 is to try to put the claimant and the poster of the statement in direct contact and leave the website operator out of the dispute. Roughly speaking, having received a complaint, if the operator is unable to contact the poster, or the poster does not respond with a plausible name and address, or if they ask for the statement to be taken down, then the operator must take the statement down, otherwise the operator need only pass on the poster's contact details and may then leave the statement up without worrying about it.

In practice, section 5 is quite involved. Hopefully the following summary will help.

The process starts when someone (the ``complainant'') complains to the operator that a statement on their website is defamatory of the complainant. In order to constitute a proper ``notice of complaint'' the complainant must send a notice containing all the following information:

\begin{itemize}
\item their electronic mail address
\item the meaning which the complainant attributes to the statement referred to
\item which aspects of the statement the complainant believes are:
  \begin{itemize}
  \item factually inaccurate; or
  \item opinions not supported by fact
  \end{itemize}
\item a confirmation that the complainant does not have sufficient information
\item whether the complainant consents to the operator providing the poster wit
\end{itemize}

Some of this is potentially useful. In my experience many complaints of defamation do not make clear what exactly is wrong with a statement, or in some cases fail to identity a specific statement at all. Forcing the complainant to set this out in a notice of complaint is useful.

An oddity about the notice is that the complainant has to confirm that they do not have sufficient information to bring proceedings, but the test in section 5 is that it is not possible for them to identify the poster. As discussed under the section 10 offence above, these are really quite different situations.

Even if the complainant fails to set out all the information they are supposed to, the operator must still respond within 48 hours of receiving the notice from the complainant, informing the complainant that:

\begin{itemize}
\item the notice does not comply with the requirements set out in section 5(6)(a) to (c) of the Act and regulation 2; and
\item what those requirements are
\end{itemize}

Another oddity here is that the operator does not have to say what is wrong with the notice. As far as I can tell, a standard form response to any defective notice of complaint is fine. In practice it would almost certainly be better to tell the complainant where you think they have gone wrong.

If a valid notice of complaint is received, the next step seems to be designed to check whether the poster is a repeat offender. If

\begin{itemize}
\item that complainant has sent two or more previous notices of complaint about a statement that:
  \begin{itemize}
  \item was posted on the same website
  \item by the same person
  \item and conveys the same or substantially the same imputation as each of the previous notices
  \end{itemize}
\item on each of those occasions the statement was removed from the website in accordance with the regulations (which would not be the case if the poster had formally resisted removal, or the statement was removed for some other reason)
\item the complainant informs the operator, at the same time as sending the notice of complaint (you would expect, in the same letter or email) that the complainant has sent a notice of complaint to the operator on two or more previous occqasions in relation to the statement
\end{itemize}

If all these conditions are met, the operator must remove the statement within 48 hours of receiving the notice of complaint and inform the complainant that they have done so.

Normally, a notice of complaint will not relate to earlier complaints. In that case, the operator must decide whether they are able to contact the poster personally. If they are not, then they must, within 48 hours of receiving the notice, remove the statement and send an acknowledgement to the complainant informing them that they have removed the statement.

``Able to contact'' means able to contact {\bf electronically}. If the poster is sitting in the same office as the operator, but the operator does not have an email address etc for them, then they are not ``able to contact'' them and must remove the statement.

If the operator is able to contact the poster, they must then, within 48 hours, send an acknowledgement to the complainant informing them that the operator is contacting the poster and also send the poster:

\begin{itemize}
\item a copy of the notice of complaint (with the complainant's name and address removed if they have not consented to the supply of that information to the poster)
\item notification that the statement may be removed unless:
  \begin{itemize}
  \item the operator receives a response from the poster by midnight on the 5th day after the day on which the notification was sent; and
  \item the poster's response contains:
    \begin{itemize}
    \item the poster's full name
    \item the postal address of the poster's home or business
    \item whether the poster consents to the operator providing the previous two items of information to the complainant
    \end{itemize}
  \end{itemize}
\item notification that the poster's name and address will not be released to the complainant unless the poster consents or the operator is ordered to do so by a court
\end{itemize}

If, by midnight on the 5th day, the poster has not responded, or if they have responded but their response failed to contain the requirement information and indications as to whether the the poster consents to its release, of if they have responded correctly and asked for the statement to be removed, the operator must, within 48 hours, remove the statement from the website and send a notice to the complainant indicating that they have done so. 

If a reasonable operator would consider the name or postal address in the poster's response to be ``obviously false'', then the poster's response is to be treated as if it did not contain that information.

If the poster responds in time, with all the required information, and states that they wish the statement to remain, then the operator does not have to remove it. However they must, within 48 hours, inform the complainant that the poster does not wish the statement to be removed and that it has not been removed. If the poster consented to their name or address being provided to the complainant, that must be supplied at the same time. If the poster did not conset, the complainant must be notified in writing of that fact.

\subsection{E-commerce directive}

The \href{http://eur-lex.europa.eu/legal-content/en/ALL/?uri=CELEX:32000L0031}{e-commerce directive} creates a general defence for those who are only ``hosting'' information, against almost all forms of liability for that information
including defamation, but also other things such as copyright. The main exception is liability under the data protection directive (and hence the Data Protection Act 1998). 

The defence applies to someone hosting information if:

\begin{itemize}
\item
  they are an Information Society Service
\item
  they are innocent, in that:

  \begin{itemize}
  \item
    they had no no actual knowledge of the unlawful information or
  \item
    they acted expeditiously to remove or disable access once they had actual knowledge
  \end{itemize}
\end{itemize}

Furthermore a hosting provider has no obligation to search out potential libels. For example in  \emph{Metropolitan International Schools Ltd}, google could not  be  required to prevent future defamatory snippets appearing as a result
  of any search (although this decision would probably have been reached
  even without the benefit of the directive).

The defence does not prevent an injunction being made against a hosting provider.

\subsection{Identity of claimant}

The claimant needs to be identifiable, but it is enough if some readers
are able to identify the claimant from the information given. They do
not have to be named directly.

\begin{itemize}
\item
  ``{the man who lives in that house is a paedophile''
  \hyperref[sdfootnote7sym]{\textsuperscript{7}}}
\item
  ``{X is illegimate''\hyperref[sdfootnote8sym]{\textsuperscript{8}}}
\item
  Hulton v Jones is a case of (probably) ``accidental'' defamation,
  which extends potential liablility but it is unclear how far.
\item
  O'Shea v MGN a case where someone looked like the person photographed
  - creates considerable difficulties if the idea were followed widely.
\item
  class libel - although it is possible to libel a class of people a
  claimant has to prove that they individually were intended as the
  target because the courts accept that people make loose and wide
  generalisations

  \begin{itemize}
  \item
    ``{all lawyers are thieves'' would not permit any lawyer to
    sue\hyperref[sdfootnote9sym]{\textsuperscript{9}}}
  \item
    Knupffer v London Express Newspaper was a case where the class of
    people (members of an organisation) was very small, but the claim
    still failed for lack of proof that a particular individual was
    intended.
  \end{itemize}
\end{itemize}

Practical example for WDTK: officers of a public body are criticised
either by title ``the Freedom of Information Officer'', or by
implication ``whoever was handling this case was incompetent''.

\subsection{public bodies}

{The House of Lords held in Derbyshire County Council v Times Newspapers
that an organ of local or central government may not sue for defamation.
This probably applies to other public bodies, but how far the line is
drawn is unclear. Happily the principle applies to political parties as
well\hyperref[sdfootnote10sym]{\textsuperscript{10}} but we do not have
the wonderful US rule which makes it very difficult for public figures
to sue.}

Something defamatory of a council may easily imply that someone in the
council is defamed too. The fact that someone is a public servant may sue if they are defamed even if the defamation is linked to their carrying out public functions.\footnote{McLaughlin v London Borough of Lambet [2010] EWHC 2726}

Corporations in general do have reputations and can sue for libel. This
applies even to very large and powerful
corporations\hyperref[sdfootnote11sym]{\textsuperscript{11}} although if
they are unable to show any trading loss their damages will be kept in
``tight bounds''.\hyperref[sdfootnote12sym]{\textsuperscript{12}}

\subsection{publication}

Must be to someone other than the claimant and/or defendant (so
insulting someone in an email sent only to them is not defamatory). Two
authorities are relevant to us:

\begin{itemize}
\item
  Byrne v Deane - allowing something to remain that you could remove may
  constitute publication
\item
  Godfrey v Demon Internet - for example leaving defamatory posts you
  have been informed about on a USENET server would constitute
  publication.
\end{itemize}

\paragraph{Linking?}

Does linking to material constitute publication? There is (as yet) no
direct authority on this point in English law. Some old cases could give
us a clue:

\begin{itemize}
\item
  Hird v Wood - the defendant sat on a stool near a placard which had
  been put up on the roadway containing defamatory matter. He remained
  there for a long time smoking a pipe and he continually pointed at the
  placard with his finger and thereby attracted to it the attention of
  all who passed by. HELD: publication.
\item
  Smith v Wood - the defendant has a copy of a libellous caricature
  print. A witness heard that the defendant had a copy, went to visit
  him and asked to see it. The defendant produced it to him and pointed
  out the figure of the plaintiff. HELD: no sufficient evidence of
  publication.
\item
  Lawrence v Newberry - letter published in a newspaper referred its
  readers to a speech in the House of Lords which was alleged to contain
  defamatory matter. HELD: the letter published the defamatory matter in
  the speech.
\end{itemize}

It is hard to draw any firm conclusions. It seems to me that a link to
something defamatory with text saying ``here is a defamation'' would
certainly be publication.

\appendix
%\renewcommand{\thesection}{Appendix \Alph{section}}
\titleformat{\section}[display]{\bf\Large}{\huge Appendix \Alph{section}}{1pt}{}{}
\setcounter{secnumdepth}{1}
\section{Case Extracts}

\subsection{Byrne v Deane}

The plaintiff was a member of a golf club of which the two defendants
were the proprietors and the female defendant also the secretary. One of
the rules of the club provided that ``no notice or placard shall be
posted in the club premises without the consent of the secretary.''
Certain automatic gambling machines had been kept by the defendants on
the club premises for the use of the members of the club. Some one gave
information to the police, with the result that the machines were
removed from the club premises. On the following day some one put up on
the wall of the club a typewritten paper containing the following
verse:-

"For many years upon this spot\\You heard the sound of a merry
bell\\Those who were rash and those who were not\\Lost and made a spot
of cash\\But he who gave the game away\\May he byrnn in hell and rue the
day."

The word ``byrnn'' was blacked out in the original and the word ``burn''
substituted for it. The plaintiff brought a libel action against the two
defendants alleging that they had published the words in the notice of
and concerning him to the members of the club. He alleged that the words
meant that he had reported to the police the existence of the machines
upon the club premises and that he had been guilty of underhand
disloyalty to the members of the club.

Held: (1) not libellous because a law-abiding member of society would
think Mr Byrne had acted properly; (2) there was sufficient evidence of
publication because the defendants had not removed the note despite
being aware of it.

\subsection{\texorpdfstring{{{Capital and Counties Bank v George Henty
\&
Sons\hyperref[sdfootnote13sym]{\textsuperscript{13}}}}}{Capital and Counties Bank v George Henty \& Sons13}}

H. \& Sons were in the habit of receiving, in payment from their
customers, cheques on various branches of a bank, which the bank cashed
for the convenience of H. \& Sons at a particular branch. Having had a
squabble with the manager of that branch H. \& Sons sent a printed
circular to a large number of their customers (who knew nothing of the
squabble)---``H. \& Sons hereby give notice that they will not receive
in payment cheques drawn on any of the branches of the'' bank. The
circular became known to other persons; there was a run on the bank and
loss inflicted. The bank having brought an action against H. \& Sons for
libel, with an innuendo that the circular imputed insolvency.

Held: the natural meaning of the words was not libellous.

\subsection{\texorpdfstring{{{Hulton v
Jones\hyperref[sdfootnote14sym]{\textsuperscript{14}}}}}{Hulton v Jones14}}

Mr Artemus Jones, a barrister in practice, had been at one time on the
staff of the Sunday Chronicle , a newspaper owned and published by the
appellants, and contributed articles signed by himself to some of the
appellants' publications. The appellants published in the Sunday
Chronicle an article a part of which ran thus;

````Upon the terrace marches the world, attracted by the motor races---a
world immensely pleased with itself, and minded to draw a wealth of
inspiration---and, incidentally, of golden cocktails---from any scheme
to speed the passing hour. \ldots{} `Whist! there is Artemus Jones with
a woman who is not his wife, who must be, you know---the other thing!'
whispers a fair neighbour of mine excitedly into her bosom friend's ear.
Really, is it not surprising how certain of our fellow-countrymen behave
when they come abroad? Who would suppose, by his goings on, that he was
a churchwarden at Peckham? No one, indeed, would assume that Jones in
the atmosphere of London would take on so austere a job as the duties of
a churchwarden. Here, in the atmosphere of Dieppe, on the French side of
the Channel, he is the life and soul of a gay little band that haunts
the Casino and turns night into day, besides betraying a most unholy
delight in the society of female butterflies.''

Artemus Jones complained and the newspaper printed the following
apology:

``It seems hardly necessary for us to state that the imaginary Mr.
Artemus Jones referred to in our article was not Mr. Thomas Artemus
Jones, barrister, but, as he has complained to us, we gladly publish
this paragraph in order to remove any possible misunderstanding and to
satisfy Mr. Thomas Artemus Jones we had no intention whatsoever of
referring to him.''

Note that: ``apart from the name used, none of the details with regard
to the imaginary personage described in the libel were applicable to the
plaintiff, inasmuch as he is not a married man, nor a churchwarden, nor
a resident of Peckham; nor was he either a frequenter of Dieppe or there
at the time when the scene described in the alleged libel took place.''

Held: for the plaintiff.

Chase v News Group Newspapers

On the front page of the newspaper for June 22 there was a large
headline ``Nurse is probed over 18 deaths: World Exclusive''. The
article said that she was suspected of overdosing terminally ill
``youngsters'' with painkillers. It identified the children concerned as
nine boys and nine girls, aged between eight weeks and 17 years.

Held: The words were incapable of meaning merely ``reasonable grounds to
investigate''. In order to justify a publication to the effect that
there were reasonable grounds to suspect that the claimant was guilty of
an offence, a defendant had to establish that there were objectively
reasonable grounds for such suspicion. The defendant could not establish
this and would therefore fail.

\section{Libel v Slander}
\label{sec:app-libelvslander}

Recall that originally libel was for writing and slander for spoken forms of defamation. Over time libel has absorbed almost any form of non-spoken defamation, including anything broadcast and anything recorded in permanent form.

Why does this matter? Libel has a historical origin as a part of the
criminal process and a means by which the state could prevent scurrilous
printed material. As a result libel is one of the very small number of
causes of action where there was no need to prove damage of any kind.

Slander, on the other hand, has a different background. In most cases
for there to be a valid claim for slander, actual damage would have to
be proved. There were some exceptions to this for particularly serious
slanders, such as a a statement that a woman was unchaste.\footnote{\href{http://www.legislation.gov.uk/ukpga/Vict/54-55/51/section/1}{s1, Slander of Women Act 1891}, repealed by \href{http://www.legislation.gov.uk/ukpga/2013/26/section/14/enacted}{s14}, Defamation Act 2013}

Although it may feel like a slander, everything written on the web will
be treated as a libel. Thus whether or not there was damage is
immaterial.

Since the Defamation Act 2013 has imposed a requirement of ``serious harm''\footnote{s1} it seems to me that this distinction is likely to be very obscure. Claims for slander are fairly rare. The fact that a defamatory statement was slander rather than libel would matter only where there had as yet been no damage to the claimant's reputation but where serious harm was likely to be caused.


\hyperdef{}{sdfootnote1}{}
\hyperref[sdfootnote1anc]{1}Players familiar with pre-6th edition
\emph{Magic: the Gathering} will find applications reminiscent of
interrupts.

\hyperdef{}{sdfootnote2}{}
\hyperref[sdfootnote2anc]{2}Libel Act 1792 (32 Geo. III c. 60)

\hyperdef{}{sdfootnote3}{}
\hyperref[sdfootnote3anc]{3}\emph{Fox v Goodfellow} (1926) NZLR 58

\hyperdef{}{sdfootnote4}{}
\hyperref[sdfootnote4anc]{4}Lord Justice Scrutton in \emph{Bennison v
Julton}{, The Times, April 13, 1926}

\hyperdef{}{sdfootnote5}{}
\emph{\hyperref[sdfootnote5anc]{5}Rubber Improvement Ltd v Daily
Telegraph}{ {[}1964{]} A.C. 234}

\hyperdef{}{sdfootnote6}{}
\hyperref[sdfootnote6anc]{6}\emph{Chase v News Group Newspapers}
{[}2003{]} EMLR 11

\hyperdef{}{sdfootnote7}{}
\hyperref[sdfootnote7anc]{7}Channel 7 Sydney v Parras {[}2002{]} NSWCA
202

\hyperdef{}{sdfootnote8}{}
\hyperref[sdfootnote8anc]{8}Cassidy v Daily Mirror

\hyperdef{}{sdfootnote9}{}
\hyperref[sdfootnote9anc]{9}Eastwood v Holmes

\hyperdef{}{sdfootnote10}{}
\hyperref[sdfootnote10anc]{10}Goldsmith v Bhoyrul {[}1998{]} Q.B. 459

\hyperdef{}{sdfootnote11}{}
\hyperref[sdfootnote11anc]{11}MacDonald's Corporation v Steel (No. 4),
the Independent, May 10, 1999

\hyperdef{}{sdfootnote12}{}
\hyperref[sdfootnote12anc]{12}Jameel v Wall Street Journal Europe SPRL
(No.3), {[}2006{]} UKHL 44; {[}2007{]} 1 A.C. 359

\hyperdef{}{sdfootnote13}{}
\hyperref[sdfootnote13anc]{13}(1881-82) L.R. 7 App. Cas. 741

\hyperdef{}{sdfootnote14}{}
\hyperref[sdfootnote14anc]{14}{[}1910{]} A.C. 20

\end{document}

%%  LocalWords:  Derbyshire Godfrey
