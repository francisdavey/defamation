\documentclass[ignorenonframetext,]{beamer}
\setbeamertemplate{caption}[numbered]
\setbeamertemplate{caption label separator}{:}
\setbeamercolor{caption name}{fg=normal text.fg}
\usepackage{amssymb,amsmath}

\usetheme{FD}
\usepackage{fixltx2e} % provides \textsubscript
\usepackage{lmodern}

\usepackage{fontspec,xltxtra,xunicode}
\defaultfontfeatures{Mapping=tex-text,Scale=MatchLowercase}

\usepackage[normalem]{ulem}

\newcommand{\euro}{€}

% use upquote if available, for straight quotes in verbatim environments
\IfFileExists{upquote.sty}{\usepackage{upquote}}{}
% use microtype if available
\IfFileExists{microtype.sty}{\usepackage{microtype}}{}

% Comment these out if you don't want a slide with just the
% part/section/subsection/subsubsection title:
\AtBeginPart{
  \let\insertpartnumber\relax
  \let\partname\relax
  \frame{\partpage}
}
\AtBeginSection{
  \let\insertsectionnumber\relax
  \let\sectionname\relax
  \frame{\sectionpage}
}
\AtBeginSubsection{
  \let\insertsubsectionnumber\relax
  \let\subsectionname\relax
  \frame{\subsectionpage}
}

\setlength{\parindent}{0pt}
\setlength{\parskip}{6pt plus 2pt minus 1pt}
\setlength{\emergencystretch}{3em}  % prevent overfull lines
\setcounter{secnumdepth}{0}

\title{Staying protected \\
how to avoid lawsuits in the age of user-generated content}
\date{}

\begin{document}
\frame{\titlepage}

\frametitle{Menu}

\begin{itemize}
\item  I. Law of Defamation

  \begin{itemize}
  \item    terminology
  \item    (some) anatomy of defamation claims
  \item    what is defamatory?
  \item    defences
  \end{itemize}
\item  II. Defences for website operators

  \begin{itemize}
  \item    special defences
  \item    claimants
  \item    publication and hyperlinking
  \item    government consultation on web archives
  \end{itemize}
\item Questions: any time
\end{itemize}

\end{frame}

\begin{frame}
\frametitle{Defamation: the problem}
\begin{itemize}
\item ``Defamation'' - publishing a statement that harms someone's reputation (literally ``reduces their fame'')
\item Dangerous features:
  \begin{itemize}
  \item claimant does not need to prove falsity
  \item may be done ``innocently''
  \item applies to every repetition
  \item \sout{trial by jury}
  \end{itemize}
\end{itemize}
\end{frame}

\begin{frame}
  \frametitle{Alternative threats}
  \begin{itemize}
  \item Malicious falsehood
    \begin{itemize}
    \item claimant must prove falsity
    \item requires ``malice''
    \end{itemize}
  \item Data Protection Act 1998
    \begin{itemize}
    \item new regulation ``soon''
    \item EU-wide
    \end{itemize}
  \item Privacy rights
  \item Intellectual property rights
  \end{itemize}
\end{frame}

\begin{frame}
  \frametitle{Defamation: unusual features}
  \begin{itemize}
  \item {\bf Long} history of development by the courts
  \item Trial by jury
  \item Human Rights Act 1998
  \item Defamation Act 2013
    \begin{itemize}
    \item from: 1st January 2014
    \item few cases so far
    \item messy patch
    \end{itemize}
  \end{itemize}
\end{frame}

\begin{frame}
\frametitle{Terminology}

\begin{itemize}
\item ``Defamation'' = libel + slander
  \begin{itemize}
  \item slander = spoken
  \item libel = almost everything else
  \end{itemize}
\item Defamatory statements may be true
\item Law = law of England and Wales
\end{itemize}
\end{frame}

\begin{frame}
\frametitle{Elements of defamation}

\begin{itemize}
\item  A statement S
\item  Identifiably about C
\item  S means M
\item  M is defamatory of C
\item  S was published by D
\end{itemize}

\end{frame}

\begin{frame}
\frametitle{What is a defamatory meaning?}

\begin{itemize}
\item  ``one to the claimant's discredit''
\item  ``tends to lower him in the estimation of others''
\item  ``causes him to be shunned or avoided''
\item  ``exposes him to hatred, contempt or ridicule''
\item  United States (restatement):

  {}``it tends to harm the reputation of another so as to lower him or
  her in the estimation of the community or deter third parties from
  associating or dealing with him or her''
\end{itemize}

~


\end{frame}

\begin{frame}
\frametitle{Defamatory meaning II}

\begin{itemize}
\item  {Likely} not {actual} effect of meaning

  \begin{itemize}
  \item    proof of actual effect irrelevant
  \item    can still be defamatory even if no-one who heard it believed
    it
  \end{itemize}
\item  Effect is on ``right thinking persons generally''

  \begin{itemize}
  \item    Byrne v Deane
  \end{itemize}
\end{itemize}

~


\end{frame}

\begin{frame}
\frametitle{Are these defamatory?}

\begin{itemize}
\item  C is insane

  C has HIV

  C has been raped

  C has/had heart disease

  C is illegitimate

  C has leprosy

  X is a better journalist than C

  C is a lawyer of only average ability
\end{itemize}

~


\end{frame}

\begin{frame}
\frametitle{Meaning I}

\begin{itemize}
\item  ``natural and ordinary meaning''
\item  includes inferences

  \begin{itemize}
  \item    ``have you heard that Fox was reported twice as a spy?''
  \item    covers many news reporting situations and WDTK
  \end{itemize}
\item  not strained, forced or utterly unreasonable

  \begin{itemize}
  \item    not enough that someone {might}{ understand it that way}
  \item    {Capital and Counties Bank v Henty}
  \item    {``}{chop and tomato sauce''}
  \end{itemize}
\end{itemize}

~


\end{frame}

\begin{frame}
\frametitle{Meaning II: the reader}

\begin{itemize}
\item  \begin{itemize}
  \item    ordinary reasonable fair-minded reader

    \begin{itemize}
    \item      {--}may be guilty of a certain amount of loose-thinking
    \item      {--}does not read a sensational article with cautious and critical
      care
    \item      {--}goes by broad impression
    \item      {--}does not construe words as would a lawyer
    \end{itemize}
  \item    of reasonable intelligence
  \item    with an ordinary person's general knowledge
  \end{itemize}
\end{itemize}

~


\end{frame}

\begin{frame}
\frametitle{Meaning III: innuendos}

\begin{itemize}
\item  True or ``legal'' innuendo

  \begin{itemize}
  \item    in the light of additional facts which may not be general
    knowledge
  \item    a separate cause of action
  \item    can make a meaning defamatory or not defamatory
  \end{itemize}
\end{itemize}

~


\end{frame}

\begin{frame}
\frametitle{Defences}

\begin{itemize}
\item  Truth
\item  Privilege
\item  Qualified privilege
\item  Public Interest
\item  Honest Opinion
\item  Offer of amends
\end{itemize}

~


\end{frame}

\begin{frame}
\frametitle{Truth}

\begin{itemize}
\item  Truth a complete defence to civil suit
\item  May plead an alternative ({Lucas-Box}{) meaning}
\item  Repetition rule

  \begin{itemize}
  \item    Lewis v Daily Telegraph
  \item    adding ``allegedly'' no good
  \item    giving right to reply or flagging something as possibly
    unreliable is also useless
  \end{itemize}
\end{itemize}

~


\end{frame}

\begin{frame}
\frametitle{Justification II}

\begin{itemize}
\item  Investigations of wrongdoing

  \begin{itemize}
  \item    ``Police arrested C for child sexual abuse yesterday''
  \end{itemize}
\item  {Chase} meanings:

  \begin{itemize}
  \item    was guilty
  \item    reasonable grounds to suspect
  \item    reasonable grounds to investigate
  \end{itemize}
\end{itemize}

~


\end{frame}

\begin{frame}
\frametitle{Qualified Privilege}

\begin{itemize}
\item  Co-ordination of duty and interest

  \begin{itemize}
  \item    Duty or interest in publication
  \item    Duty or interest in receipt
  \end{itemize}
\item  Defeated by {malice}
\item  Examples

  \begin{itemize}
  \item    confidential references
  \item    communications amongst the team
  \item    newspaper reporting? {Reynolds}
  \end{itemize}
\end{itemize}

~


\end{frame}

\begin{frame}
\frametitle{Fair comment}

\begin{itemize}
\item  Comment not statement

  \begin{itemize}
  \item    may be an inference
  \end{itemize}
\item  Honest -- defeated by {malice }
\item  {Matter of public interest}
\item  {Comment on existing facts}

  \begin{itemize}
  \item    {facts must be true}
  \end{itemize}
\end{itemize}

~


\end{frame}

\begin{frame}
\frametitle{Outcomes}

\begin{itemize}
\item  Injunction

  \begin{itemize}
  \item    may be made in the interim
  \item    balance of convenience test for interim injunctions
  \end{itemize}
\item  Damages

  \begin{itemize}
  \item    potentially very large
  \item    decided by juries, controlled by the court of appeal
  \end{itemize}
\item  Costs
\end{itemize}

~


\end{frame}

\begin{frame}
\frametitle{Offer of amends}

\begin{itemize}
\item  An offer to

  \begin{itemize}
  \item    make and publish

    \begin{itemize}
    \item      {--}a suitable correction; and
    \item      {--}a sufficient apology
    \end{itemize}
  \item    {pay }{compensation to be agreed or determined}
  \end{itemize}
\item  {Plus}

  \begin{itemize}
  \item    {not an admission of liability}
  \item    {acceptance prevents future claim}
  \end{itemize}
\item  {Minus}

  \begin{itemize}
  \item    {only useful for innocent defamations}
  \item    {may not use another defence}
  \end{itemize}
\end{itemize}

~


\end{frame}

\begin{frame}
\frametitle{II. Particular issues for websites}
\end{frame}

\begin{frame}
\frametitle{Special defences I}

\begin{itemize}
\item  Defamation Act 1996

  \begin{itemize}
  \item    not the author, editor or publisher
  \item    took reasonable care in respect of publication
  \item    did not know and had no reason to know that D caused or
    contributed to the publication of a defamatory statement
  \end{itemize}
\end{itemize}

~


\end{frame}

\begin{frame}
\frametitle{Special Defences II}

\begin{itemize}
\item  E-commerce directive ``hosting''

  \begin{itemize}
  \item    all information liability
  \item    applies only to Information Society Services
  \end{itemize}
\item  Absolute defence provided

  \begin{itemize}
  \item    no actual knowledge; or
  \item    acts expeditiously to remove or disable
  \end{itemize}
\item  {No need to search out potential libels}

  \begin{itemize}
  \item    {Metropolitan International Schools Ltd}
  \end{itemize}
\item  Does not prevent injunctions
\end{itemize}

~


\end{frame}

\begin{frame}
\frametitle{Identity of claimant}

\begin{itemize}
\item  Must be able to identify the claimant

  \begin{itemize}
  \item    ``the man who lives in that house is a paedophile''

    ``X is illegimate''
  \end{itemize}
\item  Accidents

  \begin{itemize}
  \item    Hulton v Jones 
  \item    O'Shea v MGN
  \end{itemize}
\item  class libel 

  \begin{itemize}
  \item    ``all lawyers are thieves''
  \item    Knupffer v London Express Newspaper
  \end{itemize}
\end{itemize}

~


\end{frame}

\begin{frame}
\frametitle{Who can sue?}

\begin{itemize}
\item  Everybody, except

  \begin{itemize}
  \item    Dead people
  \item    Public bodies?

    \begin{itemize}
    \item      {--}{Derbyshire}{ -- arms of local and central government}
    \item      {--}{Goldsmith}{ -- applies to political parties as well}
    \item      {--}{beware of defaming individuals via the public body}
    \end{itemize}
  \end{itemize}
\item  {Corporations can sue}

  \begin{itemize}
  \item    {unless they have no trading reputation within the
    jurisdiction}
  \end{itemize}
\end{itemize}

~


\end{frame}

\begin{frame}
\frametitle{Publication}

\begin{itemize}
\item  Making available on a server is enough

  \begin{itemize}
  \item    Byrne v Deane
  \item    Godfrey v Demon Internet
  \end{itemize}
\item  Indirect publication

  \begin{itemize}
  \item    Hird v Wood
  \item    Lawrence v Newberry
  \item    Smith v Wood
  \end{itemize}
\item  Linking?
\end{itemize}

~


\end{frame}

\begin{frame}
\frametitle{Consultation}

\begin{itemize}
\item  Limitation Act 1980

  \begin{itemize}
  \item    one year time limit
  \item    subject to possible extension
  \end{itemize}
\item  Multiple publication rule

  \begin{itemize}
  \item    Loutchansky
  \end{itemize}
\item  Consultation

  \begin{itemize}
  \item    driven by Rupert Murdoch press
  \item    response by 16 December 2009
  \end{itemize}
\end{itemize}

~

\end{frame}

\end{document}

\end{frame}

