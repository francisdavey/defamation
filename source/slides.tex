\documentclass[ignorenonframetext,]{beamer}
\setbeamertemplate{caption}[numbered]
\setbeamertemplate{caption label separator}{:}
\setbeamercolor{caption name}{fg=normal text.fg}
\usepackage{amssymb,amsmath}

\usetheme{FD}
\usepackage{fixltx2e} % provides \textsubscript
\usepackage{lmodern}

\usepackage{fontspec,xltxtra,xunicode}
\defaultfontfeatures{Mapping=tex-text,Scale=MatchLowercase}

\usepackage[normalem]{ulem}

\newcommand{\euro}{€}

% use upquote if available, for straight quotes in verbatim environments
\IfFileExists{upquote.sty}{\usepackage{upquote}}{}
% use microtype if available
\IfFileExists{microtype.sty}{\usepackage{microtype}}{}

% Comment these out if you don't want a slide with just the
% part/section/subsection/subsubsection title:
\AtBeginPart{
  \let\insertpartnumber\relax
  \let\partname\relax
  \frame{\partpage}
}
\AtBeginSection{
  \let\insertsectionnumber\relax
  \let\sectionname\relax
  \frame{\sectionpage}
}
\AtBeginSubsection{
  \let\insertsubsectionnumber\relax
  \let\subsectionname\relax
  \frame{\subsectionpage}
}

\setlength{\parindent}{0pt}
\setlength{\parskip}{6pt plus 2pt minus 1pt}
\setlength{\emergencystretch}{3em}  % prevent overfull lines
\setcounter{secnumdepth}{0}

\title{Staying protected \\
how to avoid lawsuits in the age of user-generated content}
\date{}

\begin{document}
\frame{\titlepage}

\begin{frame}
\frametitle{Overview}
\begin{itemize}
\item  I. Law of Defamation

  \begin{itemize}
  \item introduction
  \item terminology
  \item what is defamatory?
  \item defences
  \item outcomes
  \end{itemize}
\item  II. Internet related issues

  \begin{itemize}
  \item  special defences
  \item  publication and hyperlinking
  \item  miscellany
  \end{itemize}
\item  III. Practicalities
\end{itemize}

\end{frame}

\begin{frame}
  \frametitle{Questions}
  \begin{itemize}
  \item Interrupt at any time
  \item Please
    \begin{itemize}
    \item questions relevant to place in the talk
    \item don't anticipate the end of the story
    \item {\bf do not} ask about specific cases (``I've been sent this letter...'')
    \end{itemize}
  \end{itemize}
\end{frame}

\begin{frame}
\frametitle{Defamation: the problem}
\begin{itemize}
\item ``Defamation'' - publishing a statement that harms someone's reputation (literally ``reduces their fame'')
\item Dangerous features:
  \begin{itemize}
  \item claimant does not need to prove falsity
  \item may be done ``innocently''
  \item applies to every repetition
  \item tried in the High Court by default
  \item \sout{trial by jury}
  \end{itemize}
\end{itemize}
\end{frame}

\begin{frame}
  \frametitle{Alternative threats}
  \begin{itemize}
  \item Malicious falsehood
    \begin{itemize}
    \item claimant must prove falsity
    \item requires ``malice''
    \end{itemize}
  \item Data Protection Act 1998
    \begin{itemize}
    \item new regulation ``soon''
    \item EU-wide
    \end{itemize}
  \item Privacy rights
  \item Intellectual property rights
  \end{itemize}
\end{frame}

\begin{frame}
  \frametitle{Defamation: unusual features}
  \begin{itemize}
  \item {\bf Long} history of development by the courts
  \item Trial by jury
  \item Human Rights Act 1998
  \item Defamation Act 2013
    \begin{itemize}
    \item from: 1st January 2014
    \item few cases so far
    \item messy patch
    \end{itemize}
  \end{itemize}
\end{frame}

\begin{frame}
\frametitle{Terminology}

\begin{itemize}
\item ``Defamation'' = libel + slander
  \begin{itemize}
  \item slander = spoken
  \item libel = almost everything else
  \end{itemize}
\item Defamatory statements may be true
\item Law = law of England and Wales
\item Publish = communication to a third party
\end{itemize}
\end{frame}

\begin{frame}
\frametitle{Elements of defamation}

\begin{itemize}
\item  A statement S
\item  Identifiably about C
\item  S means M
\item  M is defamatory of C
\item  S was published by D to a third party
\item  NEW: S caused or is likely to cause ``serious harm'' to C
\end{itemize}

\end{frame}

\begin{frame}
\frametitle{What is a defamatory meaning?}

\begin{itemize}
\item ``Words [that] tend to lower the plaintiff in the estimation of right-thinking members of society generally''
\item  ``Causes him to be shunned or avoided''
\item  ``Exposes him to hatred, contempt or ridicule''
\item  United States (restatement):

    ``it tends to harm the reputation of another so as to lower him or
    her in the estimation of the community or deter third parties from
    associating or dealing with him or her''

\end{itemize}

\end{frame}

\begin{frame}
\frametitle{Defamatory meaning II}

\begin{itemize}
\item  {Likely} not {actual} effect of meaning

  \begin{itemize}
  \item    proof of actual effect irrelevant
  \item    can still be defamatory even if no-one who heard it believed
    it
  \end{itemize}
\item  Effect is on ``right thinking persons generally''

  \begin{itemize}
  \item    Byrne v Deane
  \end{itemize}
\end{itemize}

~


\end{frame}

\begin{frame}
\frametitle{Are these defamatory?}
  C is insane

  C has HIV

  C has been raped

  C has/had heart disease

  C is illegitimate

  C has leprosy

  X is a better journalist than C

  C is a lawyer of only average ability

\end{frame}

\begin{frame}
\frametitle{Meaning I}

\begin{itemize}
\item  ``Natural and ordinary meaning''
\item  Includes inferences
  \begin{itemize}
  \item    ``have you heard that Fox was reported twice as a spy?''
  \item    covers many news reporting situations and WDTK
  \end{itemize}
\item  Not strained, forced or utterly unreasonable
  \begin{itemize}
  \item  not enough that someone {might}{ understand it that way}
  \item  {\it Capital and Counties Bank v Henty}
  \item  {``}{chop and tomato sauce''}
  \end{itemize}
\end{itemize}

\end{frame}

\begin{frame}
\frametitle{Meaning II: the reader}


\begin{itemize}
  \item  Ordinary reasonable fair-minded reader

    \begin{itemize}
    \item may be guilty of a certain amount of loose-thinking
    \item does not read a sensational article with cautious and critical care
    \item goes by broad impression
    \item does not construe words as would a lawyer
    \end{itemize}
  \item  Of reasonable intelligence
  \item  With an ordinary person's general knowledge
  \end{itemize}


\end{frame}

\begin{frame}
\frametitle{Meaning III: innuendos}

\begin{itemize}
\item  True or ``legal'' innuendo

  \begin{itemize}
  \item in the light of additional facts which may not be general knowledge
  \item a separate cause of action
  \item can make a meaning defamatory or not defamatory
  \end{itemize}
\end{itemize}
\end{frame}

\begin{frame}
  \frametitle{Serious Harm}
  \begin{itemize}
  \item NEW section 1 Defamation Act 2013
  \item Claimant will have to prove the statement
    \begin{itemize}
    \item has caused serious harm
    \item is likely to cause serious harm
    \end{itemize}
  \item {\it Cooke v MGN}
  \item Harm will often be inferred (eg alleging C is a terrorist)
  \item Body that ``trades for profit''
    \begin{itemize}
    \item seroius financial loss
    \end{itemize}
  \end{itemize}
\end{frame}

\begin{frame}
\frametitle{Defences}

\begin{itemize}
\item  Truth
\item  Privilege
\item  Qualified privilege
\item  Public Interest
\item  Honest Opinion
\item  Offer of amends
\end{itemize}
\end{frame}

\begin{frame}
\frametitle{Truth}

\begin{itemize}
\item  Truth a complete defence to civil suit
\item  Repetition rule

  \begin{itemize}
  \item    Lewis v Daily Telegraph
  \item    adding ``allegedly'' no good
  \item    giving right to reply or flagging something as possibly
    unreliable is also useless
  \end{itemize}
\end{itemize}
\end{frame}

\begin{frame}
\frametitle{Justification II}

\begin{itemize}
\item  Investigations of wrongdoing

  \begin{itemize}
  \item    ``Police arrested C for child sexual abuse yesterday''
  \end{itemize}
\item  {Chase} meanings:

  \begin{itemize}
  \item    was guilty
  \item    reasonable grounds to suspect
  \item    reasonable grounds to investigate
  \end{itemize}
\end{itemize}
\end{frame}

\begin{frame}
\frametitle{Qualified Privilege}

\begin{itemize}
\item  Co-ordination of duty and interest

  \begin{itemize}
  \item    Duty or interest in publication
  \item    Duty or interest in receipt
  \end{itemize}
\item  Defeated by {malice}
\item  Examples

  \begin{itemize}
  \item    confidential references
  \item    communications amongst the team
  \item    newspaper reporting? {Reynolds}
  \end{itemize}
\end{itemize}
\end{frame}

\begin{frame}
\frametitle{Honest Opinion: the defence}

\begin{itemize}
\item The statement was a statement of opinion 
\item The basis of that opinion was indicated
\item An honest person could have held the opinion, on the basis of
  \begin{itemize}
  \item any fact which existed at the time of publication
  \item anything asserted in a privileged statement
    \begin{itemize}
    \item matter of public interest
    \item peer-reviewed statement in scientific or academic journal
    \item various privileged reports (eg court proceedings)
    \end{itemize}
  \end{itemize}
\item Old law: ``Fair Comment'' or ``Honest Comment'' (abolished)
\end{itemize}
\end{frame}


\begin{frame}
  \frametitle{Honest Opinion: defeating}
  \begin{itemize}
  \item C can defeat defence if they prove
    \begin{itemize}
    \item D published the statement and did not hold the opinion
    \item D published someone else's statement who did not hold the opinion and D:
      \begin{itemize}
      \item knew
      \item ought to have known
      \end{itemize}
    \end{itemize}
  \item Old law: ``malice''
  \end{itemize}

\end{frame}

\begin{frame}
\frametitle{Offer of amends}

\begin{itemize}
\item  An offer to

  \begin{itemize}
  \item    make and publish

    \begin{itemize}
    \item a suitable correction; and
    \item a sufficient apology
    \end{itemize}
  \item    {pay }{compensation to be agreed or determined}
  \end{itemize}
\item  {Plus}

  \begin{itemize}
  \item    {not an admission of liability}
  \item    {acceptance prevents future claim}
  \end{itemize}
\item  {Minus}

  \begin{itemize}
  \item    {only useful for innocent defamations}
  \item    {may not use another defence}
  \end{itemize}
\end{itemize}

\end{frame}

\begin{frame}
\frametitle{Outcomes}

\begin{itemize}
\item  Injunction
  \begin{itemize}
  \item very unlikely to be  made in the interim
  \item may be made against innocent website owner
  \end{itemize}
\item  Damages

  \begin{itemize}
  \item historically very large
  \item court of appeal reigned in jury awards
  \item still very large
  \end{itemize}
\item  Costs
\end{itemize}
\end{frame}


\begin{frame}
\frametitle{II. Particular issues for websites}
\begin{itemize}
\item Preliminary
  \begin{itemize}
  \item is there publication?
  \item innocent dissemination
  \end{itemize}
\item General
  \begin{itemize}
  \item Section 1, Defamation Act 1996
  \item Section 10, Defamation Act 2013
  \end{itemize}
\item Technology specific
  \begin{itemize}
  \item Section 5, Defamation Act 2013
  \item E-commerce directive
  \end{itemize}
\end{itemize}
\end{frame}


\begin{frame}
\frametitle{Publication}

\begin{itemize}
\item Common law - knowledge not required
  \begin{itemize}
  \item printer
  \item printer's servant (who ``clapped down the press'')
  \item newspaper vendor
  \end{itemize}
\item {\it Bunt v Tiley}
  \begin{itemize}
  \item {\bf passive} ISP not a publisher at all 
  \item probably correct
  \end{itemize}
\end{itemize}

\end{frame}


\begin{frame}
\frametitle{Common law publication - without notice}
  \begin{itemize}
    \item {\it Byrne v Deane}
      \begin{itemize}
      \item notice left on clubroom wall
      \end{itemize}
    \item {\it Godfrey v Demon Internet}
      \begin{itemize}
      \item USENET posting
      \item publisher even though unaware of the post
      \end{itemize}
    \item {\it Tamiz v Google}
      \begin{itemize}
      \item blogger platform
      \item not a publisher when unaware
      \end{itemize}
    \item {\it Metropolitan International Schools v Designtechnica}
      \begin{itemize}
      \item Google not a publisher of snippets
      \end{itemize}
    \item Knowledge appears to make you a publisher
  \end{itemize}
 
%\item  Indirect publication
\end{frame}


\begin{frame}
  \frametitle{Innocent dissemination}
  \begin{itemize}
  \item Available to ``secondary'' publishers
  \item Elements of the defence:
    \begin{itemize}
    \item D did not know the publication:
      \begin{itemize}
      \item contained a libel
      \item was of a kind likely to contain a libel
      \end{itemize}
    \item Lack of knowledge was not due to negligence
    \end{itemize}
  \item May avoid pre-notice liability (but it depends)
  \item Mostly superseded by other defences
  \end{itemize}
\end{frame}

\begin{frame}
\frametitle{Defamation Act 1996, section 1}

\begin{itemize}
\item Not the ``author'', ``editor'' or ``publisher''
  \begin{itemize}
  \item Author = originator
  \item Editor, may include pre-moderation
  \item Publisher = commercial publisher
    \begin{itemize}
    \item {\it McGrath v Dawkins}
    \item Amazon not a commercial publisher of its website
    \end{itemize}
  \end{itemize}
\item Took reasonable care in respect of publication
\item Did not know and had no reason to know that D caused or
  contributed to the publication of a defamatory statement
\end{itemize}

\end{frame}

\begin{frame}
  \frametitle{Defamation Act 2013, section 10}
  \begin{itemize}
  \item Not the ``author'', ``editor'' or ``publisher''
  \item Not reasonably practicable for to sue the author editor or publisher
    \begin{itemize}
    \item does it matter that sueing would be useless?
    \item what about suing anonymous posters?
    \end{itemize}
  \end{itemize}
\end{frame}

\begin{frame}
  \begin{quote}
     The person or persons who have offered the publishers of the Sun, the Daily Mail, and the Daily Mirror newspapers a copy of the book 'Harry Potter and the order of the Phoenix' by JKRowling or any part thereof and the person or persons who has or have physical possession of a copy of the said book or any part thereof without the consent of the claimants.
  \end{quote}
\end{frame}

\begin{frame}
  \frametitle{Defamation Act 2013, section 5}
  \begin{itemize}
  \item An {\it optional} defence
  \item The idea:
    \begin{itemize}
    \item complaint is made about a posted statement on a {\bf website}
    \item the website operator contacts the poster
    \item if the poster wants the post to stay up, it stays
    \item poster and complainant are put in touch
    \item otherwise it is removed
    \end{itemize}
  \item The reality: (!?!?)
  \end{itemize}
\end{frame}

\begin{frame}
  \frametitle{Section 5: the defence}
  \begin{itemize}
  \item D is an Operator of a website who did not post the statement
  \item Claimant must prove:
    \begin{itemize}
    \item it was not possible for them to identify the person who posted the statement
    \item they gave a ``notice of complaint'' to the operator (or were deemed to have given one)
    \item the operator failed to respond to the notice correctly
    \end{itemize}
  \end{itemize}
\end{frame}

\begin{frame}
  \frametitle{Section 5: Notice of Complaint}

  \begin{itemize}
  \item The complainant's electronic mail address
  \item The meaning which the complainant attributes to the statement referred to
  \item Which aspects of the statement the complainant believes are:
    \begin{itemize}
    \item factually inaccurate; or
    \item opinions not supported by fact
    \end{itemize}
  \item A confirmation that the complainant does not have sufficient information about the poster to bring proceedings against them
  \item Whether the complainant consents to the operator providing the poster with the complainant's name and email address
\end{itemize}

\end{frame}

\begin{frame}
  \frametitle{Section 5: operator action}

  \begin{itemize}
  \item Time limit: 48 hours
  \item Defective notice, inform the complainant that:
\begin{itemize}
\item the notice does not comply with the requirements set out in section 5(6)(a) to (c) of the Act and regulation 2; and
\item what those requirements are
\end{itemize}
\item If able to contact the poster {\bf electronically}
  \begin{itemize}
  \item send information to poster
  \item inform complainant
  \end{itemize}
\item If not able to contact the poster
  \begin{itemize}
  \item remove statement
  \item inform complainant
  \end{itemize}

  \end{itemize}
\end{frame}

\begin{frame}
  \frametitle{Section 5: poster's response}
  \begin{itemize}
  \item Time limit 48 hours from response
  \item Statement must be removed if:
    \begin{itemize}
    \item no response by midnight on day 5
    \item response is defective
      \begin{itemize}
      \item includes ``obviously false'' name or postal address
      \end{itemize}
    \item poster asks for the statement to be removed
    \end{itemize}
  \item Statement is retained if poster responds correctly and asks for it to stay
  \item Inform complainant
    \begin{itemize}
    \item that the statement is staying or going
    \item of the poster's name and address if the poster consented and asked for the statement to stay
    \end{itemize}
  \end{itemize}
\end{frame}

%Could put in something about vexatious posters

\begin{frame}
  \frametitle{Section 5: repeat offenders}
\begin{itemize}
\item Complainant informs the operator
  \begin{itemize}
  \item at the same time as sending the notice
  \item complainant has sent notice on 2+ occasion ``in relation to the statement''
  \end{itemize}
\item Two or more previous notices of complaint about a statement that:
  \begin{itemize}
  \item was posted on the same website
  \item by the same person
  \item conveys the same/substantially the same imputation as each previous notice
  \end{itemize}
\item Each time statement was removed from the website
  \begin{itemize}
  \item in accordance with the regulations
  \end{itemize}
\end{itemize}

\end{frame}

\begin{frame}
\frametitle{E-commerce directive}

\begin{itemize}
\item  Articles 12 - 15
  \begin{itemize}
  \item mere conduit
  \item caching
  \item {\bf hosting}
  \end{itemize}

  \item Almost all forms of liability
    \begin{itemize}
    \item key exception: data protection
    \end{itemize}
  \item Applies only to ``information society services''
  \item Does not prevent injunctions

\end{itemize}

\end{frame}

\begin{frame}
  \frametitle{Information Society Services}
  \begin{itemize}
  \item A service
    \begin{itemize}
    \item normally provided for remuneration
    \item at a distance
    \item by means of electronic equipment for processing and storing data
    \item at the individual request of a recipient of the service
    \end{itemize}
  \item Remuneration
    \begin{itemize}
    \item could be provided by advertisements
    \item need not be provided by the users
    \item must represent some form of economic activity
    \end{itemize}
  \item Where does this leave civil society?
  \end{itemize}
\end{frame}

\begin{frame}
  \frametitle{E-commerce directive: hosting}
  \begin{itemize}
  \item ``Hosting'':
    \begin{itemize}
    \item storage of information
    \item provided by a recipient of the service (i.e. not the service provider)
    \item recipient is not under hoster's control
   \end{itemize}
  \item Absolute defence for ``hosting'' provided the hoster
    \begin{itemize}
    \item has no actual knowledge; or
    \item acts expeditiously to remove or disable access
    \end{itemize}
  \item  {No need to search out potential libels}
  \end{itemize}
  
\end{frame}

% See Gatley 6.44 Kashcke v Gray point

\begin{frame}
\frametitle{Hyperlinks}
\begin{itemize}
\item Old cases
  \begin{itemize}
  \item Hird v Wood
  \item Lawrence v Newberry
  \item Smith v Wood
  \end{itemize}
\item Newer cases
  \begin{itemize}
  \item Crookes v Newton (Canada)
    \begin{itemize}
    \item no liability
    \item difference of views on context
    \end{itemize}
  \item {\it Budu v BBC} - linked material provides context
  \item {\it Islam Expo v Spectator (1828)} - incorporated hyperlinked information into the text
  \item {\it McGrath v Dawkins} - home button might create responsibility for linked site
  \end{itemize}
\end{itemize}
\end{frame}

\begin{frame}
\frametitle{Identity of claimant}

\begin{itemize}
\item  Must be able to identify the claimant

  \begin{itemize}
  \item    ``the man who lives in that house is a paedophile''

    ``X is illegitimate''
  \end{itemize}
\item  Accidents

  \begin{itemize}
  \item    Hulton v Jones 
  \item    O'Shea v MGN
  \end{itemize}
\item  class libel 

  \begin{itemize}
  \item    ``all lawyers are thieves''
  \item    Knupffer v London Express Newspaper
  \end{itemize}
\end{itemize}

~


\end{frame}

\begin{frame}
\frametitle{Who can sue?}


  \begin{itemize}
  \item Not Dead people
  \item Public bodies?

    \begin{itemize}
    \item {\it Derbyshire v Times Newspapers}{ -- arms of local and central government}
    \item {\it Goldsmith}{ -- applies to political parties as well}
    \item {\it Duke v University of Salford} -- but not universities
    \item {\it McLaughlin v Lambeth} -- or public servants
    \end{itemize}
  \end{itemize}
\item Corporations
  \begin{itemize}
  \item unless they have no trading reputation within the
    jurisdiction
  \end{itemize}

\end{frame}

\begin{frame}
\frametitle{Single publication}

\begin{itemize}
\item  Limitation Act 1980
  \begin{itemize}
  \item  one year time limit
  \item  subject to possible extension
  \end{itemize}
\item  Old: multiple publication rule
  \begin{itemize}
  \item continuous availability = continuous publication
  \item Loutchansky notices
  \end{itemize}
\item  Reform: single publication rule
  \begin{itemize}
  \item driven by Rupert Murdoch press
  \item section 8 Defamatoin Act 2013
  \end{itemize}
\end{itemize}

\end{frame}

\begin{frame}
  \frametitle{Single publication}
  \begin{itemize}
  \item First publication of statement
  \item Subsequent publication
    \begin{itemize}
    \item includes ``substantially the same'' statement
    \end{itemize}
  \item Cause of action ``accrues'' on first publication
  \item Not if manner of publication is ``materially different'', taking into account
    \begin{itemize}
    \item level of prominence
    \item extent of subsequent publication
    \end{itemize}
  \item Court can still extend time
  \end{itemize}
\end{frame}

\begin{frame}
  \frametitle{Practical action}
  \begin{itemize}
  \item Decide on a strategy
  \item Decide whether to use section 5
  \item Design a response procedure
  \item Designate individuals to deal with responses
  \item Designate senior individuals to make ``risky'' decisions
  \end{itemize}
\end{frame}

\begin{frame}
  \frametitle{Example procedure}
  \begin{itemize}
  \item Is the complaint clear?
    \begin{itemize}
    \item if not - respond to complainant asking for more information
    \end{itemize}
  \item Does the complaint hold any water?
    \begin{itemize}
    \item if unclear - refer to more senior person
    \item if still unclear - consult lawyer
    \item if not, keep statement and respond to complainant with explanation
    \end{itemize}
  \item Would it be practical for the complainant to sue the author?
    \begin{itemize}
    \item if so, refer complainant to section 10, Defamation Act 2013
    \item explain the situation to the complainant
    \end{itemize}
  \item Do we keep the statement anyway?
  \end{itemize}

\end{frame}

\end{document}



